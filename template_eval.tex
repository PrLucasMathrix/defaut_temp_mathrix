\documentclass[12pt]{article}

%initialisation des packages%


\usepackage[T1]{fontenc}
\usepackage[utf8]{inputenc}
\usepackage[french]{babel}
\usepackage{fourier}
\usepackage{textcomp} 
\usepackage{colortbl}
\usepackage[T1]{fontenc}
\usepackage[utf8]{inputenc}
\usepackage{fourier}
\usepackage[scaled=0.875]{helvet}
\renewcommand{\ttdefault}{lmtt}
\usepackage{amsmath,amssymb}
\usepackage{fancybox}
\usepackage[normalem]{ulem}
\usepackage{lscape}
\usepackage{diagbox}
\usepackage{colortbl}
\usepackage{pst-plot,pst-tree,pstricks,pst-node,pst-text}
\usepackage{multirow}
\usepackage{mathrsfs}
\usepackage{hyperref}
\usepackage{textcomp} 
\usepackage{enumerate}
\usepackage{tabularx}
\usepackage{fancyhdr}
\usepackage[top=2.5cm,bottom=2.5cm, left=3cm, right=3cm,a4paper]{geometry}
\usepackage{tabularx}
%fin initialisation des packages%

%insertion de nouvelle commandes%
\newcommand{\euro}{\eurologo{}}

\renewcommand{\ttdefault}{lmtt}
\renewcommand{\thesubsection}{\textcolor{blue}{\textsc{Exercice} \arabic{subsection}}} 
\newcommand{\easy}[2]{\hspace*{0.5cm} \textbf{( ! )}\\}
\newcommand{\med}[2]{\hspace*{0.5cm} \textbf{( !! )}\\}
\newcommand{\hard}[2]{\hspace*{0.5cm} \textbf{( !!! )}\\}
%fininsertion de nouvelle commandes%

%écriture rapide des ensembles (doit être inclus dans un $$ exemple $\R$.
\newcommand{\R}{\mathbb{R}} %Réel 
\newcommand{\C}{\mathbb{C}} %Complexe
\newcommand{\Z}{\mathbb{Z}} %Relatif
\newcommand{\D}{\mathbb{D}} %Décimal
\newcommand{\Q}{\mathbb{Q}} %Rationnel 
\newcommand{\N}{\mathbb{N}} %Naturel
\newcommand{\E}{\mathbb{E}} %Espérance Mathématiques
\renewcommand{\P}{\mathbb{P}} %Proba

%Début du document% 
\begin{document}
\setlength\parindent{0mm}

	%en tête, pied de page%
		\rhead{\textbf{www.mathrix.fr}}
		\lhead{\small Prépa BAC }
		\lfoot{\small{\href{http://www.mathrix.fr}{\underline{www.mathrix.fr}}}}
		\rfoot{\small{2017-2018}}
		\renewcommand \footrulewidth{.2pt}
		\pagestyle{fancy}
		\thispagestyle{empty} 
	%fin en tête, pied de page%

	%première page%
		\begin{center} 
		{\Large{\textbf{\textsc{\decofourleft~Préparation au Baccalauréat général ~\decofourright\\[5pt]
		\vspace{0.3cm}
		SESSION 2017-2018}}}}


		\vspace{1cm}

		\shadowbox{\LARGE{\textbf{\textsc{Mathématiques - Série Scientifique}}}}\\

		\vspace{1.2cm}

		\shadowbox{\LARGE{\textbf{\textsc{Exercice sur les suites numériques}}}}\\
		\vspace{0.4cm}

		\vspace{0.4cm}
		\textbf{Les exercices sont classés par niveau, \\ 
		! : Facile\\
		!! : Moyen\\
		!!! : Difficile }
		\vspace{1.5cm}

		Le sujet est composé de 4 exercices indépendants.\\
		Le candidat doit traiter tous les exercices.\\
		La candidat est invité à faire figurer sur la copie toute trace de recherche, même incomplète ou 			non fructueuse, qu'il aura développée.

		\vspace{1.5cm}

		Avant de composer, le candidat s'assurera que le sujet comporte bien X pages.

		\vspace{1cm}

		\textbf{\textit{Programme : TERMINALE SCIENTIFIQUE 2012}}
	
		\vspace{2cm}
		\href{http://www.mathrix.fr}{\Large\underline{www.mathrix.fr}}

	%fin première page%

\newpage
	\end{center}





%EXERCICE 1


\subsection{\textcolor{blue}{\hfill 7 points}}
\label{courbe}

\medskip
\noindent\textbf{(Commun à tous les candidats) \hard }
\medskip
On considère $\forall n \in \mathbb{N}$ . \\

\begin{enumerate}

\item Soit la suite $U_{n}=3 \times q^{n}$.

\begin{enumerate}

\item Démontrer que la suite $U_{n}$ est géométrique de raison $q$ et de premier terme $U_{0}$. 

\end{enumerate}

\item Déterminer par le calcul tensoriel la relation $A \bigotimes B \cong \beta$.

\item $\dfrac{exp (-4)}{2}$

$\P$ \\
$\E$
\end{enumerate}


\medskip


\end{document}

